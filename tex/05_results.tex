% Example runs with chosen parameters
% How well does the algorithm actually work?
\section{Results}

% Intro til at vi kører forskellige set-ups 

\subsection{True parameters for underlying model}
For each of the runs, the underlying model used to generate the data was constant and is included below:


\begin{align}
    & A = 
    \begin{bmatrix}
        0.95 & 0.05 \\
        0.1  & 0.9
    \end{bmatrix} \\
    & B =
    \begin{bmatrix}
        1.0/6 & 1.0/6 & 1.0/6 & 1.0/6 & 1.0/6 & 1.0/6\\
        1.0/10 &  1.0/10 & 1.0/10 & 1.0/10 & 1.0/10 & 5.0/10
    \end{bmatrix}\\
    & \pi = 
    \begin{bmatrix}
        1/2 \\
        1/2
    \end{bmatrix}
\end{align}


\subsection{Run 1}
\textbf{Run parameters:} \\
\begin{table}[]
    \begin{tabular}{@{}ll@{}}
        \toprule
        Random seed:                            & 1   \\ \midrule
        models trained:                         & 7   \\
        Number of training iterations pr. model & 500 \\
        Number of training                      & 300 \\
        Number of test sequences                & 100 \\
        Lenght of sequences                     & 50  \\ \bottomrule
    \end{tabular}
    \caption{}
    \label{tab:my-table}
\end{table}

%figID = 1150




\subsection{Run 2}
%np.random.seed(324)
%runs = 7 # number of models trained
%training_iters = 500 #Number of training iterations pr. model
%n_train_seq = 300 # Number of generated simulated sequences
%n_test_seq = 100
%len_sim_seq = 50

%fig id: 1346

Min MSE/ min likelihood
(array([[0.144, 0.087, 0.087, 0.082, 0.088, 0.511],
        [0.165, 0.158, 0.161, 0.168, 0.162, 0.186]]),
 array([[0.871, 0.102],
        [0.039, 0.945]]),
 array([0.39716069, 0.60283931]))

ensemble
(array([[0.15257143, 0.11614286, 0.11728571, 0.11685714, 0.11828571,
         0.379     ],
        [0.64214286, 0.06771429, 0.069     , 0.072     , 0.06942857,
         0.07971429]]),
 array([[0.93328571, 0.04371429],
        [0.30528571, 0.405     ]]),
 array([0.7416403, 0.2583597]))


\subsection{Run 3}

\textbf{Run parameters:} \\
\begin{table}[]
    \begin{tabular}{@{}ll@{}}
        \toprule
        Random seed:                            & 1   \\ \midrule
        models trained:                         & 7   \\
        Number of training iterations pr. model & 250 \\
        Number of training                      & 100 \\
        Number of test sequences                & 30 \\
        Lenght of sequences                     & 50  \\ \bottomrule
    \end{tabular}
    \caption{}
    \label{tab:my-table}
\end{table}

par for min log like model: n6
(array([[0.161, 0.155, 0.156, 0.147, 0.159, 0.222],
        [0.153, 0.055, 0.069, 0.052, 0.055, 0.615]]),
 array([[0.972, 0.013],
        [0.123, 0.824]]),
 array([0.63774319, 0.36225681]))\\

par for ensemble
(array([[0.49      , 0.07428571, 0.07671429, 0.07328571, 0.076     ,
         0.20971429],
        [0.15714286, 0.10614286, 0.11357143, 0.10071429, 0.10842857,
         0.41357143]]),
 array([[0.89185714, 0.02314286],
        [0.47457143, 0.492     ]]),
 array([0.32506948, 0.67493052]))

% run id: 1302

\subsection{Run 4}
%np.random.seed(324)
%runs = 7 # number of models trained
%training_iters = 250 #Number of training iterations pr. model
%n_train_seq = 100 # Number of generated simulated sequences
%n_test_seq = 30
%len_sim_seq = 50

% run id 1324
best MSE (model 1):
(array([[0.158, 0.162, 0.162, 0.173, 0.157, 0.189],
        [0.152, 0.075, 0.088, 0.062, 0.096, 0.525]]),
 array([[0.939, 0.045],
        [0.141, 0.828]]),
 array([0.60032667, 0.39967333]))

best likelihood: (model 7)
(array([[1.   , 0.   , 0.   , 0.   , 0.   , 0.   ],
        [0.156, 0.14 , 0.143, 0.145, 0.142, 0.274]]),
 array([[0.   , 0.516],
        [0.   , 0.98 ]]),
 array([1.97702238e-18, 1.00000000e+00]))

ensemble:
(array([[0.52028571, 0.05285714, 0.08657143, 0.06957143, 0.05271429,
         0.21757143],
        [0.15471429, 0.14157143, 0.13357143, 0.13857143, 0.14642857,
         0.28514286]]),
 array([[0.62514286, 0.153     ],
        [0.41114286, 0.56871429]]),
 array([0.24619173, 0.75380827]))


\subsection{Approximation of true parameters}


\subsection{Negative Log likelihood}



% sammenlign MSE og LL

